\documentclass[11pt,a4paper]{article}


\usepackage[a4paper,nohead,margin=2.5cm]{geometry}
\usepackage{amsmath,amssymb,latexsym,amsthm,epsfig,stmaryrd}
\usepackage{german}
\usepackage{enumerate}
\usepackage[mathscr]{eucal}
\usepackage[latin1]{inputenc}

\newcommand{\h}{\mathcal{H}}
\newcommand{\F}{\mathcal{F}}
\newcommand{\Q}{\mathbb{Q}}
\newcommand{\R}{\mathbb{R}}
\newcommand{\C}{\mathbb{C}}
\newcommand{\Z}{\mathbb{Z}}
\newcommand{\N}{\mathbb{N}}
\newcommand{\Oc}{\mathbb{O}}
\newcommand{\mf}{\mathfrak}
\newcommand{\mb}{\mathbf}
\newcommand{\mc}{\mathcal}
\newcommand{\s}{\mathcal{S}}

\newcommand{\bca}{\begin{cases}}
\newcommand{\eca}{\end{cases}}
\newcommand{\beq}{\begin{equation}}
\newcommand{\eeq}{\end{equation}}
\newcommand{\bpm}{\begin{pmatrix}}
\newcommand{\epm}{\end{pmatrix}}
\newcommand{\bal}{\begin{align}}
\newcommand{\eal}{\end{align}}

\DeclareMathOperator{\supp}{supp}

\newcounter{aufgabennum}
\setcounter{aufgabennum}{22}
\newcommand{\aufgabe}[1]{\subsubsection*{Aufgabe \theaufgabennum:\refstepcounter{aufgabennum}\ \ignorespaces #1}}

\long\def\ignorethis#1{}
\parindent0cm 

\begin{document}

\pagestyle{empty}

\mbox{}\vspace{-1.8cm}
\noindent    

{\Large\bfseries
�bungen zu Analysis 1}\\
Universit�t Regensburg, Wintersemester 2017/18\\
Prof.~Dr.~F. Finster, M. Akman
\vspace{2mm}
\hrule
\vspace{5mm}
\begin{center}
	{\large\bfseries �bungsblatt 6}\\
	\vspace{2mm}
	{Abgabe: Freitag 01.12.2017 bis 12:00 Uhr}
\end{center}
\bigskip
Im Folgenden seien $(a_n)_{n\in\N}$ und $(b_n)_{n\in\N}$ Folgen reeller (oder komplexer) Zahlen.
\aufgabe{Wurzelkriterium (4 Punkte)} Beweisen Sie die folgenden Aussagen:
\begin{itemize}
	\item[i)] Gibt es ein $q\in\R$, $0<q<1$, und ein $N\in\N$, so dass f�r alle $n\geq N$ gilt
	$$\sqrt[n]{|a_n|}\leq q\,,$$
	dann konvergiert die Reihe $\sum_{n=1}^\infty a_n$ absolut.\\
	\textit{Hinweis:} Verwenden Sie das Majorantenkriterium.
	\item[ii)] Gilt $\sqrt[n]{|a_n|}\geq1$ f�r unendlich viele $n\in\N$, dann divergiert die Reihe $\sum_{n=1}^\infty a_n$.
	\item[iii)] Zeigen Sie durch ein Beispiel, dass f�r $\lim_{n\rightarrow\infty}\sqrt[n]{|a_n|}=1$ keine Aussage �ber die Konvergenz der Reihe $\sum_{n=1}^\infty a_n$ gemacht werden kann.\\
	\textit{Hinweis:} Ohne Beweis darf $\lim_{n\rightarrow\infty}\sqrt[n]{n}=1$ verwendet werden.
\end{itemize} 

\aufgabe{(4 Punkte)} �berpr�fen Sie, ob die folgenden Reihen konvergieren: 
\begin{itemize}
	\item[i)] $$\sum_{n=0}^\infty \left(\frac{n^2}{2n^2+n+1}\right)^n$$
	\item[ii)] $$\sum_{n=1}^\infty \frac{n!}{n^n}$$
	\item[iii)] $$\sum_{n=1}^\infty \left(1-\frac{1}{n^2}\right)^n$$
	\item[iv)] $$\sum_{n=2}^\infty \frac{2^{n+1}}{3\cdot 5^n}$$
\end{itemize}
Berechnen Sie den Wert der Reihe in Teil iv).

\aufgabe{(2 Punkte)} 
Die Reihe $\sum_{n=1}^\infty a_n$ sei absolut konvergent. Zeigen Sie, dass dann auch die Reihe $\sum_{n=1}^\infty a_n^2$ absolut konvergiert.
\aufgabe{Widerspruchsbeweis (2 Punkte)}
    Zeigen Sie, dass die Folge $a_n = n $ mit $n \in \N$ divergiert.

\aufgabe{(4 Punkte)} \begin{itemize}
	\item[i)] Sei $\sum_{n=1}^\infty a_n$ eine konvergente Reihe, und sei $(b_n)_{n\in\N}$ eine konvergente Folge. Konvergiert die Reihe $\sum_{n=1}^\infty a_nb_n$ ? (Beweis oder Gegenbeispiel!)
	\item[ii)] Sei nun $\sum_{n=1}^\infty a_n$ eine absolut konvergente Reihe, und sei $(b_n)_{n\in\N}$ eine konvergente Folge. Was kann man hier �ber die Konvergenz der Reihe $\sum_{n=1}^\infty a_nb_n$ sagen? (Begr�ndung!)
\end{itemize}

\end{document}
