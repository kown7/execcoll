
\documentclass[12pt]{amsart}


\textwidth 6.9in
\evensidemargin +0.01in
\oddsidemargin  -0.2in
\topmargin -0.5in \textheight 9in

\usepackage[dvips]{graphicx}
\usepackage{hyperref,color,mathdots,accents,extarrows,bigints}
\usepackage[all]{xypic}
\usepackage[ngerman]{babel}
\usepackage[ansinew]{inputenc}
\usepackage{enumitem,amsmath,amsfonts,amssymb}
\usepackage[all,graph]{xy}

\theoremstyle{plain}
\newtheorem{theorem}{Theorem}[section]
\newtheorem{lemma}[theorem]{Lemma}
\newtheorem{korollar}[theorem]{Korollar}
\newtheorem{satz}[theorem]{Satz}
\newtheorem*{satz*}{Satz}


\theoremstyle{remark}
\newtheorem*{bemerkung}{Bemerkung}
\newtheorem*{definition}{Definition}
\newtheorem*{notation}{Notation}
\newtheorem*{beispiel}{Beispiel}
\newtheorem*{beispiele}{Beispiele}
\newtheorem*{konvention}{Konvention}
\newtheorem*{behauptung}{Behauptung}
\newtheorem*{aussage}{Aussage}

\makeatletter
\newcommand*{\textlabel}[2]{%
  \edef\@currentlabel{#1}% Set target label
  \phantomsection% Correct hyper reference link
  #1\label{#2}% Print and store label
}
\makeatother

\newcommand*{\medcap}{\mathbin{\scalebox{1.5}{\ensuremath{\cap}}}}%
\newcommand*{\medcup}{\mathbin{\scalebox{1.5}{\ensuremath{\cup}}}}%
\newcommand*{\bigforall}{\mathbin{\scalebox{1.2}{\ensuremath{\forall}}}}%
\newcommand*{\bigexists}{\mathbin{\scalebox{1.2}{\ensuremath{\exists}}}}%

\newcommand{\lmcup}[2]{\mbox{\small$\displaystyle\bigcup\limits_{#1}^{#2}$}} % large medium  cup
\newcommand{\smcup}[2]{\mbox{\footnotesize$\displaystyle\bigcup\limits_{#1}^{#2}$}} % small medium cup
\newcommand{\tmcup}[2]{\mbox{$\textstyle \bigcup\limits_{#1}^{#2}$}} % tiny medium cup 
\newcommand{\ttmcup}[2]{\mbox{$\textstyle \bigcup\limits_{#1}^{#2}$}} % tiny medium sum
\newcommand{\lmcap}[2]{\mbox{\small$\displaystyle\bigcap\limits_{#1}^{#2}$}} % large medium  cap
\newcommand{\smcap}[2]{\mbox{\footnotesize$\displaystyle\bigcap\limits_{#1}^{#2}$}} % small medium cap
\newcommand{\tmcap}[2]{\mbox{$\textstyle \bigcap\limits_{#1}^{#2}$}} % tiny medium cap 

\def\bint{\textstyle \bigintssss}  
\def\bbint{\textstyle \bigintsss}
\def\bbbint{\textstyle\bigintss}
\def\smdownarrow{\hspace{0.07cm}\downarrow}
\def\smuparrow{\hspace{0.07cm}\uparrow}
\def\D{\mathbb{D}}
\def\Q{\mathbb{Q}} 
\def\F{\mathbb{F}} 
\def\Z{\mathbb{Z}} 
\def\R{\mathbb{R}} 
\def\C{\mathbb{C}}
\def\N{\mathbb{N}} 
\def\H{\mathbb{H}}


\newcommand{\tmint}[2]{\mbox{\large$\int\limits_{#1}^{#2}$}} % tiny medium sum
\newcommand{\opnormal}[1]{\operatorname{\textnormal{#1}}}

\DeclareMathAlphabet{\mathbf}{OML}{cmm}{b}{it}

\numberwithin{equation}{section}


%================= to be deleted at the end
\def\smup{\smuparrow}
\def\smdown{\smdownarrow}
\def\bnm{\begin{enumerate}} 
\def\enm{\end{enumerate}}
\def\bnml{\begin{enumerate}[leftmargin=1.2cm]}
\def\bnmlf{\begin{enumerate}[leftmargin=1.2cm,font=\normalfont]}
\def\bnmfl{\begin{enumerate}[leftmargin=1.2cm,font=\normalfont]}
\def\bnmf{\begin{enumerate}[font=\normalfont]}
\def\blue{\textcolor{blue}}
\def\red{\textcolor{red}}
\newcommand\dred{\textcolor{violet}}
\def\green{\textcolor{Green}}
\def\l{\lambda} 
\def\ll{\langle} \def\rr{\rangle}
\def\g{\gamma}  
\def\bp{\begin{pmatrix}}  \def\ep{\end{pmatrix}} 
\def\sm{\setminus}
\def\bn{\begin{enumerate}} 
\def\en{\end{enumerate}}
\def\graph{\operatorname{Graph}}
\def\ba{\begin{array}} \def\ea{\end{array}} 
\def\L{\Lambda}
\def\s{\sigma} 
\def\ti{\tilde} 
\def\QQ{\mathcal{Q}}
\def\omegati{\omegaidetilde}
\def\us{\underset}
\def\ub{\underbrace}
\def\eps{\epsilon}
\def\ol{\overline}
\def\suml{\sum\limits}
\def\endd{\end{document}}
\def\dt{\,dt}
\def\dz{\,dz}
\def\Re{\op{Re}}
\def\Im{\op{Im}}
\def\op{\operatorname}
\def\BB{\mathcal{B}}
\def\CC{\mathcal{C}}
\def\DD{\mathcal{D}}
\def\EE{\mathcal{E}}
\def\FF{\mathcal{F}}
\def\LL{\mathcal{L}}
\def\TT{\mathcal{T}}
\def\SS{\mathcal{S}}
\def\UU{\mathcal{U}}
\def\VV{\mathcal{V}}
\def\NN{\mathcal{N}}
\def\XX{\mathcal{X}}
\def\WW{\mathcal{W}}
\def\YY{\mathcal{Y}}
\def\MM{\mathcal{M}}
\def\PP{\mathcal{P}}
\def\RR{\mathcal{R}}
\def\AA{\mathcal{A}}
\def\ZZ{\mathcal{Z}}
\def\id{\op{id}}
\def\grad{\op{grad}}
\def\gram{\op{Gram}}
\def\div{\op{div}}
\def\vol{\op{Vol}}
\newcommand{\mfbox}[1]{\footnotesize{\mbox{#1}}}

\newcommand{\lmfrac}[2]{\mbox{\small$\displaystyle\frac{#1}{#2}$}}   % large medium frac
\newcommand{\smfrac}[2]{\mbox{\footnotesize$\displaystyle\frac{#1}{#2}$}} % small medium frac
\newcommand{\tmfrac}[2]{\mbox{\large$\frac{#1}{#2}$}} % tiny medium frac
\newcommand{\ttmfrac}[2]{\mbox{$\textstyle\frac{#1}{#2}$}}
\newcommand{\lmsum}[2]{\mbox{\small$\displaystyle\sum\limits_{#1}^{#2}$}} % large medium sum
\newcommand{\smsum}[2]{\mbox{\footnotesize$\displaystyle\sum\limits_{#1}^{#2}$}} % small medium sum
\newcommand{\tmsum}[2]{\mbox{$\textstyle \sum\limits_{#1}^{#2}$}} % tiny medium sum
\newcommand{\ttmsum}[2]{\mbox{\footnotesize{$\textstyle \sum\limits_{#1}^{#2}$}}} % tiny medium sum
\def\mindent{\hspace{1cm}}
\def\mindentp{\hspace{1.2cm}}
\def\angle{\sphericalangle}
\def\delabc{\Delta_{ABC}}
\def\delpqr{\Delta_{PQR}}
\def\ac{\ol{AC}}
\def\bc{\ol{BC}}
\def\ab{\ol{AB}}
\def\pq{\ol{PQ}}
\def\pr{\ol{PR}}
\def\qr{\ol{QR}}
\def\sms{\setminus}
\def\lac{\ell(\ol{AC})}
\def\lbc{\ell(\ol{BC})}
\def\lab{\ell(\ol{AB})}
\def\lpq{\ell(\ol{PQ})}
\def\lpr{\ell(\ol{PR})}
\def\lqr{\ell(\ol{QR})}
\def\E{\mathbb{E}}
\DeclareMathAlphabet{\mathbf}{OML}{cmm}{b}{it}
\def\nspace{\hspace{-0.3cm}}
\newcommand{\nbox}[1]{\nspace\mfbox{#1}}
\def\smup{\smuparrow}
\def\smdown{\smdownarrow}
\newcommand{\mraisebox}[2]{\mbox{\raisebox{#1}{$#2$}}}
\def\strahl{\overrightarrow}
\renewcommand{\smallmatrix}[4]{\Big( \hspace{-0.1cm}\mbox{\small{$\ba{cc}\hfill #1& \hfill #2 \\ \hfill #3& \hfill #4\ea$}}\hspace{-0.1cm} \Big)}
\let\emptyset\varnothing
\newcommand{\svector}[2]{\Big(\hspace{-0.1cm}\mbox{\small{$\ba{c}#1  \\ #2\ea$}} \hspace{-0.1cm}\Big)}
\def\bnal{\begin{enumerate}[label=(\alph*),leftmargin=1.2cm]}


\usepackage{tagging}
\usetag{Solution}				% show these tags
% \droptag{Solution}				% hide these tags; Options:





\begin{document}


                  %%%%%%%%%%%%%% Blattkopf %%%%%%%%%%%%%%%%%%%%%%%%%%%%
\noindent
{\sc
	Universit\"at Regensburg \hfill WS 2023/24 \\
	Fakult\"at f\"ur Mathematik \hfill Elementargeometrie \\
\begin{center}
	2.\ \"Ubungsblatt\\
\end{center}	
}
\begin{center}
	 Dr. Claudio Paganini \\
\end{center}
\vspace{0.5cm}
\medskip
%%%%%%%%%%%%%%%%%%%%%%%%%%%%%%%%%%%%%%%%%%%%%%%%%%%%%

%%%%%%%%%%%%%%%%%%%%%%%%%%%%%%%%%%%%%%%%%%%%%


\subsection*{Aufgabe 1}\mbox{}
Sei $\phi: \E \rightarrow\E $ eine winkelerhaltende Abbildung, mit der Eigenschaft, dass f\"ur zwei Punkte $P,Q\in \E$, $P\neq Q$ gilt, dass $\phi(P)=P$ und $\phi(Q)=Q$. Ist $\phi$ die Identit\"atsabbildung?


\begin{taggedblock}{Solution}
    


\smallskip 
\noindent \emph{L\"osung.}\mbox{}
Nein. Eine Spiegelung entlang der Geraden $g(P,Q)$ ist winkelerhaltend und l\"asst $g(P,Q)$ invariant, ist aber nicht die Identit\"at. 

\end{taggedblock}


\subsection*{Aufgabe 2}\mbox{}
\bnal
\item 
Es sei $g=\{ P+t\cdot v\,|\, t\in \R\}$ eine Gerade mit Richtungsvektor $v=(a,b)\in \R^2$.
Was ist der Richtungsvektor einer Gerade, welche senkrecht auf $g$ steht?
\item Wir betrachten die Gerade 
\[ \hspace{1cm} g\,\,:=\,\, \{ (2,3)+t\cdot (1,1)\,|\, t\in \R\}.\]
Was ist die Spiegelung von $P=(-1,2)$ entlang von $g$?
\enm

\begin{taggedblock}{Solution}
    


\smallskip 
\noindent \emph{L\"osung.}\mbox{}
\bnal
\item Ein m\"oglicher Richtungsvektor ist $(-b,a)$, denn $\svector{a}{b}\cdot \svector{-b}{a}=-ab+ab=0$.
\item Es folgt aus (a), dass die Gerade $h$ durch $(-1,2)$, welche senkrecht auf $g$ steht, gegeben ist durch 
$\{ (-1,2)+s\cdot (-1,1)\,|\,s\in \R\}$. Der Schnittpunkt $Q$ von $g$ und $h$ berechnet sich wie folgt:
\[ \mindent \ba{crcl} (1)&\hspace{-0.2cm} 2+t\!\cdot\!  1 &\hspace{-0.2cm}=&\hspace{-0.2cm}-1+s\!\cdot\!  (-1)\\
(2)&\hspace{-0.2cm}3+t\!\cdot\!  1&\hspace{-0.2cm}=&\hspace{-0.2cm} 2+s\!\cdot\!  1\ea \,\, \Rightarrow \,\,
\ba{crcl} (1')&\hspace{-0.2cm} t &\hspace{-0.2cm}=&\hspace{-0.2cm}-3-s\\
(2)&\hspace{-0.2cm}3+t&\hspace{-0.2cm}=&\hspace{-0.2cm} 2+ s\ea \,\, \Rightarrow 
 \,\,3+(-3-s)\,=\,2+s\,\, \Rightarrow \,\,  s\,=\,-1.
\]
Wir erhalten nun $Q=(-1,2)+1\cdot  (-1,1)=(0,1)$. Das Spiegelbild ist jetzt gegeben durch den Punkt 
$Q+(Q-P)=(0,1)+(1,-1)=(1,0)$. 
\enm



\begin{figure}[h]
\begin{center}
\input{spiegelung-explizit.pstex_t}
\caption{}\label{fig:}
\end{center}
\end{figure}
\end{taggedblock}


\subsection*{Aufgabe 3}
Es sei $A\subset \R^n$ eine Teilmenge. Wir sagen eine Abbildung $f\colon A\to A$ ist l\"angenerhaltend, wenn f\"ur alle $P,Q\in A$ gilt, dass $\|f(P)-f(Q)\|=\|P-Q\|$.
\bnal
\item Ist jede l\"angenerhaltende Abbildung injektiv?
\item Ist jede l\"angenerhaltende Abbildung surjektiv?
\enm


\begin{taggedblock}{Solution}
    



\smallskip
\noindent \emph{L\"osung.}\mbox{}
\bnal
\item Ja! Es seien $P,Q\in A$ mit  $P\ne Q$. Wir m\"ussen zeigen, dass $f(P)\ne f(Q)$.
In der Tat  gilt 
\[ \mindent \ba{rlll} \|f(P)-f(Q)\|&=\,\,\|P-Q\|&\ne\,\, 0\quad \Longrightarrow \,\, f(P)\ne f(Q).\\
&\smup&\smup\\
&\hspace{-1.6cm}\nbox{da $f$ l\"angenerhaltend}&\nbox{da $P\ne Q$}\ea\]
\item Nein! Z.B.\ betrachten wir $A=[0,\infty)$ und $f\colon A\to A$, gegeben durch $f(x)=x+1$. Die Abbildung ist l\"angenerhaltend, aber nicht surjektiv, denn $0$ liegt beispielsweise nicht im Bild von $f$. 
\enm 

\end{taggedblock}

\subsection*{Aufgabe 4}
\mbox{}
\bnal
\item Geben Sie eine geometrische Definition der Spiegelung in einem Punkt $P\in \E$.
\item Zeigen Sie, dass die Spiegelung im Ursprung $(0,0)\in \E=\R^2$ l\"angenerhaltend ist.
\item 
Wir kennen nun folgende Beispiele von l\"angenerhaltende Abbildung:
\bnml
\item[(i)] Verschiebung,
\item[(ii)] Spiegelung entlang einer Gerade,
\item[(iii)]  Drehung um einen Punkt.
\enm
Wie passen da die Punktspiegelung rein? Genauer gesagt, kann man Punkt\-spiegelungen durch Spiegelungen oder Drehungen beschreiben?
\enm

\begin{taggedblock}{Solution}
\noindent \emph{L\"osung.}\mbox{}
\bnal
\item Es sei $P\in \E$ und $Q\in \E$ ein anderer  Punkt.
Es sei $g=g(P,Q)$ die Gerade durch $P$ und $Q$. Wir definieren wir die Punktspiegelung von $Q$ als den Punkt $R$ auf $g(P,Q)$ mit den folgenden beiden Eigenschaften:
\bnm
\item[(1)] Der Punkt $R$ liegt auf der anderen Seite von $P$.
\item[(2)] Es gilt $\ell(\ol{PR})=\ell(\ol{PQ})$.
\enm
\item Eine Punktspiegelung um $P$ ist das gleiche wie eine Drehung um $P$ um den Winkel $\pi$. 
\enm
\end{taggedblock}


\subsection*{Aufgabe 5}
Die Verkn\"upfung von zwei Verschiebungen ist nat\"urlich wiederum eine Verschiebung.
\bnal
\item Ist die Verkn\"upfung von zwei Drehungen wiederum eine Drehung? Die Drehungen k\"onnen dabei um zwei verschiedene Punkte erfolgen.
\item Was kann man \"uber die Verkn\"upfung von zwei Spiegelungen entlang von zwei Geraden $g$ und $h$ sagen? 
Ist dies wiederum eine Spiegelung? Oder eine andere Art von l\"angenerhaltender Abbildung?
\enm

\begin{taggedblock}{Solution}

\noindent \emph{L\"osung.}\mbox{}
\bnal
\item Die Verkn\"upfung einer Drehung um den Punkt $P$ mit Drehwinkel $\alpha$ und einer Drehung um den Punkt $Q$ mit Drehwinkel $\beta$ ist  eine Drehung um einen (anderen) Punkt mit Drehwinkel $\alpha+\beta$, , au\ss er, wenn beide Drehwinkel $\pi$ sind, dan handelt es sich um eine Verschiebung. Wir werden das sp\"ater noch genauer diskutieren.
\item Es seien $g$ und $h$ zwei Geraden. Wir unterscheiden zwei F\"alle
\bnm
\item[(1)] Wenn sich $g$ und $h$ in genau einem Punkt $P$ schneiden, dann ist die Verkn\"upfung der zwei Spiegelungen eine Drehung um den Punkt $P$. 
Wir werden diese Aussage demn\"achst noch beweisen. 
\item[(2)] Wenn sich $g$ und $h$ nicht schneiden, dann ist die Verkn\"upfung der zwei Spiegelungen eine Verschiebung.
\enm
\enm

\end{taggedblock}

\subsection*{Aufgabe 6}
Es sei $g$ eine Gerade. Wir bezeichnen mit $s_g\colon \E\to \E$ die Spiegelung entlang von $g$. 
\bnal
\item Es sei $Q$ ein Punkt in $\E$. Vervollst\"andigen Sie folgenden Satz:  es gilt $s_g(Q)=Q$ genau dann, wenn ???????
\item Es sei $h$ eine Gerade, welche parallel zu $g$ verl\"auft. Zeigen Sie, dass $s_g(h)$ ebenfalls parallel zu $g$ verl\"auft.
\enm

\begin{taggedblock}{Solution}
\noindent \emph{L\"osung.}\mbox{}
\bnal
\item  Es gilt $s_g(Q)=Q$ genau dann, wenn $Q\in g$. 
\item  Nehmen wir an, dass $s_g(h)$ nicht  parallel zu $g$ verl\"auft. Dann besitzen $s_g(h)$ und $g$ genau einen Schnittpunkt $P$.
Da $P$ auf $g$ liegt folgt aus (a), dass $s_g(P)=P$. Wenn wir $s_g(h)$ also wieder zur\"uckspiegeln, dann sehen wir, dass $P$ ein Schnittpunkt von $g$ und $h$ ist. Also waren $g$ und $h$ nicht parallel.
\enm
\end{taggedblock}



\end{document}