\documentclass[11pt,a4paper]{article}


\usepackage[a4paper,nohead,margin=2.5cm]{geometry}
\usepackage{amsmath,amssymb,latexsym,amsthm,epsfig,stmaryrd}
\usepackage{german}
\usepackage{enumerate}
\usepackage[mathscr]{eucal}
\usepackage[latin1]{inputenc}

\newcommand{\h}{\mathcal{H}}
\newcommand{\F}{\mathcal{F}}
\newcommand{\Q}{\mathbb{Q}}
\newcommand{\R}{\mathbb{R}}
\newcommand{\C}{\mathbb{C}}
\newcommand{\Z}{\mathbb{Z}}
\newcommand{\N}{\mathbb{N}}
\newcommand{\Oc}{\mathbb{O}}
\newcommand{\mf}{\mathfrak}
\newcommand{\mb}{\mathbf}
\newcommand{\mc}{\mathcal}
\newcommand{\s}{\mathcal{S}}

\newcommand{\bca}{\begin{cases}}
\newcommand{\eca}{\end{cases}}
\newcommand{\beq}{\begin{equation}}
\newcommand{\eeq}{\end{equation}}
\newcommand{\bpm}{\begin{pmatrix}}
\newcommand{\epm}{\end{pmatrix}}
\newcommand{\bal}{\begin{align}}
\newcommand{\eal}{\end{align}}

\DeclareMathOperator{\supp}{supp}

\newcounter{aufgabennum}
\setcounter{aufgabennum}{13}
\newcommand{\aufgabe}[1]{\subsubsection*{Aufgabe \theaufgabennum:\refstepcounter{aufgabennum}\ \ignorespaces #1}}

\long\def\ignorethis#1{}
\parindent0cm 

\begin{document}

\pagestyle{empty}

\mbox{}\vspace{-1.8cm}
\noindent    

{\Large\bfseries
�bungen zu Analysis 1}\\
Universit�t Regensburg, Wintersemester 2017/18\\
Prof.~Dr.~F. Finster, M. Akman
\vspace{2mm}
\hrule
\vspace{5mm}
\begin{center}
	{\large\bfseries �bungsblatt 4}\\
	\vspace{2mm}
	{Abgabe: Freitag 17.11.2017 bis 12:00 Uhr}
\end{center}
\bigskip

\aufgabe{} Sei $(a_n)_{n\in\N}$ eine Folge von Null verschiedener komplexer Zahlen mit folgender Eigenschaft:\\
Es gibt ein $a\in\C$, so dass zu jedem $C>0$ ein $N\in\N$ existiert, so dass f�r alle $n\in\N$ mit $n>N$ gilt:
$$d(a_n,a)>C\,.$$
\begin{itemize}
	\item[i)] Zeigen Sie, dass sogar jede beliebige komplexe Zahl $z\in\C$ f�r die Folge $(a_n)_{n\in\N}$ dieselbe Eigenschaft wie die komplexe Zahl $a$ besitzt.
	\item[ii)] Existiert $\displaystyle \lim_{n\rightarrow\infty}\frac{1}{a_n}$?\\[0.3em]
	(Die Antwort ist durch Beweis oder Gegenbeispiel zu begr�nden!)
\end{itemize}

\aufgabe{} Sei $(a_n)_{n\in\N}$ eine Folge positiver reeller Zahlen mit $\lim_{n\rightarrow\infty}a_n=0$. Beweisen Sie, dass es unendlich viele Indizes $n$ gibt, so dass $a_m\leq a_n$ f�r alle $m\geq n$ gilt.

\aufgabe{Intervallschachtelung} F�r ein Intervall $I\subset\R$ der Form $[a,b]$ mit $a<b$ definiert man die \textit{L�nge} $|I|$ des Intervalls $I$ als $b-a$.\\
Eine \textit{Intervallschachtelung} ist eine Folge $I_1,I_2,I_3,\ldots$ von Intervallen der Form $I_n=[a_n,b_n]\subset\R$ mit $a_n<b_n$, die folgende Eigenschaften besitzt:
\begin{itemize}
	\item[1.] $I_{n+1}\subset I_n$ f�r $n=1,2,3,\ldots$
	\item[2.] Zu jedem $\varepsilon>0$ gibt es ein Intervall $I_n$ der L�nge $|I_n|<\varepsilon$.
\end{itemize}
Zeigen Sie, dass es zu einer Intervallschachtelung $(I_n)_{n\in\N}$ genau eine reelle Zahl $a$ gibt, so dass $a\in I_n$ f�r alle $n\in\N$.\\

\textit{Hinweis:} Beweisen Sie zuerst, dass man durch eine Intervallschachtelung zwei konvergente Folgen $(a_n)_{n\in\N}$ und $(b_n)_{n\in\N}$ erh�lt. Zeigen Sie dann, dass die jeweiligen Limites �bereinstimmen, also $\lim_{n\rightarrow\infty}a_n=\lim_{n\rightarrow\infty}b_n$.

\aufgabe{Konvergenzs�tze} Es seien $(a_n)_{n\in\N}$ und $(b_n)_{n\in\N}$ zwei komplexe        konvergente Folgen mit $\lim_{n\rightarrow\infty}a_n=a$ und $\lim_{n\rightarrow\infty}b_n=b$. Zeigen Sie:
\begin{itemize}
    \item[i)] $$\lim_{n\rightarrow\infty}(a_n+b_n)=a+b\,.$$
    \item[ii)] Im Fall~$b\neq0$ gilt
            $$\lim_{n\rightarrow\infty}\frac{1}{b_n}=\frac{1}{b} \,.$$
            \textit{Hinweis:} Beweisen Sie zun�chst, dass es ein $N\in\N$ gibt, so dass f�r   alle $n\geq N$ gilt: $|b_n|\geq \frac{1}{2}|b|$.
            \item[iii)] Im Fall~$b\neq0$ gilt weiterhin
            $$\lim_{n\rightarrow\infty}\frac{a_n}{b_n}=\frac{a}{b} \,.$$
\end{itemize}
\aufgabe{*Quantoren (Bonusaufgabe)}
\begin{itemize}
    \item[i)] Dr�cken Sie die Aussage $\exists! \,x : A(x)$ (in Worten: es gibt genau ein
    $x$ mit der Eigenschaft $A(x)$) nur mit Hilfe der Symbole
        $\exists,\forall,\land, \lor, =$ und $\neg$ aus.

    \item[ii)] Sei $(a_n)_{n \in  \N}$ eine Folge reeller Zahlen. Negieren Sie formal die folgende Aussage:
\begin{align*}
    \forall \varepsilon \in \R^+ \;\;\exists N \in \N \;\; \forall n \in \N \::\: (n \geq N
    \Rightarrow |a_n|\leq \varepsilon),
\end{align*}
wobei $\R^+$ die Menge der positiven reellen Zahlen ist und $|a|$ den Betrag
von $a$ bezeichnet. Beschreiben sie anhand einer Skizze, was die Aussage und
        ihre Negation �ber die Folge $(a_n)_{n \in \N}$ aussagen.
\end{itemize}
\textit{Die Bonusaufgabe wird nicht bewertet. Die Bearbeitung wird aber allen
empfohlen, die sich nicht sicher mit Quantoren f�hlen.}
\end{document}
