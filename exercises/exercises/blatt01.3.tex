\subsection*{Aufgabe 3}
Es sei $A\subset \R^n$ eine Teilmenge. Wir sagen eine Abbildung $f\colon A\to A$ ist l\"angenerhaltend, wenn f\"ur alle $P,Q\in A$ gilt, dass $\|f(P)-f(Q)\|=\|P-Q\|$.
\bnal
\item Ist jede l\"angenerhaltende Abbildung injektiv?
\item Ist jede l\"angenerhaltende Abbildung surjektiv?
\enm


\begin{taggedblock}{Solution}
    



\smallskip
\noindent \emph{L\"osung.}\mbox{}
\bnal
\item Ja! Es seien $P,Q\in A$ mit  $P\ne Q$. Wir m\"ussen zeigen, dass $f(P)\ne f(Q)$.
In der Tat  gilt 
\[ \mindent \ba{rlll} \|f(P)-f(Q)\|&=\,\,\|P-Q\|&\ne\,\, 0\quad \Longrightarrow \,\, f(P)\ne f(Q).\\
&\smup&\smup\\
&\hspace{-1.6cm}\nbox{da $f$ l\"angenerhaltend}&\nbox{da $P\ne Q$}\ea\]
\item Nein! Z.B.\ betrachten wir $A=[0,\infty)$ und $f\colon A\to A$, gegeben durch $f(x)=x+1$. Die Abbildung ist l\"angenerhaltend, aber nicht surjektiv, denn $0$ liegt beispielsweise nicht im Bild von $f$. 
\enm 

\end{taggedblock}
