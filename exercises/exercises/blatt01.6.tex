\subsection*{Aufgabe 6}
Es sei $g$ eine Gerade. Wir bezeichnen mit $s_g\colon \E\to \E$ die Spiegelung entlang von $g$. 
\bnal
\item Es sei $Q$ ein Punkt in $\E$. Vervollst\"andigen Sie folgenden Satz:  es gilt $s_g(Q)=Q$ genau dann, wenn ???????
\item Es sei $h$ eine Gerade, welche parallel zu $g$ verl\"auft. Zeigen Sie, dass $s_g(h)$ ebenfalls parallel zu $g$ verl\"auft.
\enm

\begin{taggedblock}{Solution}
\noindent \emph{L\"osung.}\mbox{}
\bnal
\item  Es gilt $s_g(Q)=Q$ genau dann, wenn $Q\in g$. 
\item  Nehmen wir an, dass $s_g(h)$ nicht  parallel zu $g$ verl\"auft. Dann besitzen $s_g(h)$ und $g$ genau einen Schnittpunkt $P$.
Da $P$ auf $g$ liegt folgt aus (a), dass $s_g(P)=P$. Wenn wir $s_g(h)$ also wieder zur\"uckspiegeln, dann sehen wir, dass $P$ ein Schnittpunkt von $g$ und $h$ ist. Also waren $g$ und $h$ nicht parallel.
\enm
\end{taggedblock}



