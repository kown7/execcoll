\subsection*{Aufgabe 5}
Die Verkn\"upfung von zwei Verschiebungen ist nat\"urlich wiederum eine Verschiebung.
\bnal
\item Ist die Verkn\"upfung von zwei Drehungen wiederum eine Drehung? Die Drehungen k\"onnen dabei um zwei verschiedene Punkte erfolgen.
\item Was kann man \"uber die Verkn\"upfung von zwei Spiegelungen entlang von zwei Geraden $g$ und $h$ sagen? 
Ist dies wiederum eine Spiegelung? Oder eine andere Art von l\"angenerhaltender Abbildung?
\enm

\begin{taggedblock}{Solution}

\noindent \emph{L\"osung.}\mbox{}
\bnal
\item Die Verkn\"upfung einer Drehung um den Punkt $P$ mit Drehwinkel $\alpha$ und einer Drehung um den Punkt $Q$ mit Drehwinkel $\beta$ ist  eine Drehung um einen (anderen) Punkt mit Drehwinkel $\alpha+\beta$, , au\ss er, wenn beide Drehwinkel $\pi$ sind, dan handelt es sich um eine Verschiebung. Wir werden das sp\"ater noch genauer diskutieren.
\item Es seien $g$ und $h$ zwei Geraden. Wir unterscheiden zwei F\"alle
\bnm
\item[(1)] Wenn sich $g$ und $h$ in genau einem Punkt $P$ schneiden, dann ist die Verkn\"upfung der zwei Spiegelungen eine Drehung um den Punkt $P$. 
Wir werden diese Aussage demn\"achst noch beweisen. 
\item[(2)] Wenn sich $g$ und $h$ nicht schneiden, dann ist die Verkn\"upfung der zwei Spiegelungen eine Verschiebung.
\enm
\enm

\end{taggedblock}
