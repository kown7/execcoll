\subsection*{Aufgabe 4}
\mbox{}
\bnal
\item Geben Sie eine geometrische Definition der Spiegelung in einem Punkt $P\in \E$.
\item Zeigen Sie, dass die Spiegelung im Ursprung $(0,0)\in \E=\R^2$ l\"angenerhaltend ist.
\item 
Wir kennen nun folgende Beispiele von l\"angenerhaltende Abbildung:
\bnml
\item[(i)] Verschiebung,
\item[(ii)] Spiegelung entlang einer Gerade,
\item[(iii)]  Drehung um einen Punkt.
\enm
Wie passen da die Punktspiegelung rein? Genauer gesagt, kann man Punkt\-spiegelungen durch Spiegelungen oder Drehungen beschreiben?
\enm

\begin{taggedblock}{Solution}
\noindent \emph{L\"osung.}\mbox{}
\bnal
\item Es sei $P\in \E$ und $Q\in \E$ ein anderer  Punkt.
Es sei $g=g(P,Q)$ die Gerade durch $P$ und $Q$. Wir definieren wir die Punktspiegelung von $Q$ als den Punkt $R$ auf $g(P,Q)$ mit den folgenden beiden Eigenschaften:
\bnm
\item[(1)] Der Punkt $R$ liegt auf der anderen Seite von $P$.
\item[(2)] Es gilt $\ell(\ol{PR})=\ell(\ol{PQ})$.
\enm
\item Eine Punktspiegelung um $P$ ist das gleiche wie eine Drehung um $P$ um den Winkel $\pi$. 
\enm
\end{taggedblock}

