\subsection*{Aufgabe 2}\mbox{}
\bnal
\item 
Es sei $g=\{ P+t\cdot v\,|\, t\in \R\}$ eine Gerade mit Richtungsvektor $v=(a,b)\in \R^2$.
Was ist der Richtungsvektor einer Gerade, welche senkrecht auf $g$ steht?
\item Wir betrachten die Gerade 
\[ \hspace{1cm} g\,\,:=\,\, \{ (2,3)+t\cdot (1,1)\,|\, t\in \R\}.\]
Was ist die Spiegelung von $P=(-1,2)$ entlang von $g$?
\enm

\begin{taggedblock}{Solution}
    


\smallskip 
\noindent \emph{L\"osung.}\mbox{}
\bnal
\item Ein m\"oglicher Richtungsvektor ist $(-b,a)$, denn $\svector{a}{b}\cdot \svector{-b}{a}=-ab+ab=0$.
\item Es folgt aus (a), dass die Gerade $h$ durch $(-1,2)$, welche senkrecht auf $g$ steht, gegeben ist durch 
$\{ (-1,2)+s\cdot (-1,1)\,|\,s\in \R\}$. Der Schnittpunkt $Q$ von $g$ und $h$ berechnet sich wie folgt:
\[ \mindent \ba{crcl} (1)&\hspace{-0.2cm} 2+t\!\cdot\!  1 &\hspace{-0.2cm}=&\hspace{-0.2cm}-1+s\!\cdot\!  (-1)\\
(2)&\hspace{-0.2cm}3+t\!\cdot\!  1&\hspace{-0.2cm}=&\hspace{-0.2cm} 2+s\!\cdot\!  1\ea \,\, \Rightarrow \,\,
\ba{crcl} (1')&\hspace{-0.2cm} t &\hspace{-0.2cm}=&\hspace{-0.2cm}-3-s\\
(2)&\hspace{-0.2cm}3+t&\hspace{-0.2cm}=&\hspace{-0.2cm} 2+ s\ea \,\, \Rightarrow 
 \,\,3+(-3-s)\,=\,2+s\,\, \Rightarrow \,\,  s\,=\,-1.
\]
Wir erhalten nun $Q=(-1,2)+1\cdot  (-1,1)=(0,1)$. Das Spiegelbild ist jetzt gegeben durch den Punkt 
$Q+(Q-P)=(0,1)+(1,-1)=(1,0)$. 
\enm



\begin{figure}[h]
\begin{center}
\input{spiegelung-explizit.pstex_t}
\caption{}\label{fig:}
\end{center}
\end{figure}
\end{taggedblock}

