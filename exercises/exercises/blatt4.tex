\aufgabe{*Quantoren (Bonusaufgabe)}
\begin{itemize}
    \item[i)] Drücken Sie die Aussage $\exists! \,x : A(x)$ (in Worten: es gibt genau ein
    $x$ mit der Eigenschaft $A(x)$) nur mit Hilfe der Symbole
        $\exists,\forall,\land, \lor, =$ und $\neg$ aus.

    \item[ii)] Sei $(a_n)_{n \in  \N}$ eine Folge reeller Zahlen. Negieren Sie formal die folgende Aussage:
\begin{align*}
    \forall \varepsilon \in \R^+ \;\;\exists N \in \N \;\; \forall n \in \N \::\: (n \geq N
    \Rightarrow |a_n|\leq \varepsilon),
\end{align*}
wobei $\R^+$ die Menge der positiven reellen Zahlen ist und $|a|$ den Betrag
von $a$ bezeichnet. Beschreiben sie anhand einer Skizze, was die Aussage und
        ihre Negation über die Folge $(a_n)_{n \in \N}$ aussagen.
\end{itemize}
\textit{Die Bonusaufgabe wird nicht bewertet. Die Bearbeitung wird aber allen
empfohlen, die sich nicht sicher mit Quantoren fühlen.}

