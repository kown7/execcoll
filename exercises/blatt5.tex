\documentclass[11pt,a4paper]{article}


\usepackage[a4paper,nohead,margin=2.5cm]{geometry}
\usepackage{amsmath,amssymb,latexsym,amsthm,epsfig,stmaryrd}
\usepackage{german}
\usepackage{enumerate}
\usepackage[mathscr]{eucal}
\usepackage[latin1]{inputenc}

\newcommand{\h}{\mathcal{H}}
\newcommand{\F}{\mathcal{F}}
\newcommand{\Q}{\mathbb{Q}}
\newcommand{\R}{\mathbb{R}}
\newcommand{\C}{\mathbb{C}}
\newcommand{\Z}{\mathbb{Z}}
\newcommand{\N}{\mathbb{N}}
\newcommand{\Oc}{\mathbb{O}}
\newcommand{\mf}{\mathfrak}
\newcommand{\mb}{\mathbf}
\newcommand{\mc}{\mathcal}
\newcommand{\s}{\mathcal{S}}

\newcommand{\bca}{\begin{cases}}
\newcommand{\eca}{\end{cases}}
\newcommand{\beq}{\begin{equation}}
\newcommand{\eeq}{\end{equation}}
\newcommand{\bpm}{\begin{pmatrix}}
\newcommand{\epm}{\end{pmatrix}}
\newcommand{\bal}{\begin{align}}
\newcommand{\eal}{\end{align}}

\DeclareMathOperator{\supp}{supp}

\newcounter{aufgabennum}
\setcounter{aufgabennum}{18}
\newcommand{\aufgabe}[1]{\subsubsection*{Aufgabe \theaufgabennum:\refstepcounter{aufgabennum}\ \ignorespaces #1}}

\long\def\ignorethis#1{}
\parindent0cm 

\begin{document}

\pagestyle{empty}

\mbox{}\vspace{-1.8cm}
\noindent    

{\Large\bfseries
�bungen zu Analysis 1}\\
Universit�t Regensburg, Wintersemester 2017/18\\
Prof.~Dr.~F. Finster, M. Akman
\vspace{2mm}
\hrule
\vspace{5mm}
\begin{center}
	{\large\bfseries �bungsblatt 5}\\
	\vspace{2mm}
	{Abgabe: Freitag 24.11.2017 bis 12:00 Uhr}
\end{center}
\bigskip
\aufgabe{Konvergenz}
\begin{itemize}
    \item[i)] Sei $(a_n)_{n \in \N}$ eine reelle Nullfolge und $(b_n)_{n \in
        \N}$ eine komplexe Folge mit $|b_n| \leq a_n$. Beweisen Sie, dass
        $(b_n)_{n \in \N}$ eine Nullfolge ist.
    \item[ii)] Sei $(a_n)_{n \in \N}$ die Folge $a_n = \frac{1}{p(n)^2}$ mit 
        \begin{align*} p(n) = \inf\{p \in \N | p^2 >
        n\}.\end{align*} \\
        Berechnen Sie mit Hilfe von i):\\
        \begin{align*}\lim\limits_{n \rightarrow
        \infty}{a_n}.\end{align*}\\
        Begr�nden Sie Ihre Antwort.

    \item[iii)] Sei $(a_n)_{n \in \N}$ eine Nullfolge. Au�erdem sei $(b_n)_{n \in
\N}$ eine Folge mit $b_n \neq 0 \quad \forall n \in \N$, f�r die $0$ kein
H�ufungspunkt ist. Zeigen Sie, dass \\
        \begin{align*} \lim\limits_{n \rightarrow \infty}{\frac{a_n}{b_n}
         = 0}. \end{align*}
\end{itemize} 

\aufgabe{} Untersuchen Sie die Folgen auf Konvergenz, und bestimmen Sie gegebenenfalls den Grenzwert:
\begin{itemize}
	\item[i)] $$\left(\frac{3n^7-4n^2+11}{2n^6-3n+2}\right)_{n\in\N}\:;$$
	\item[ii)] $$\left(\frac{2n^5-12n^3+3}{3n^5-8n^2+6}\right)_{n\in\N}\:;$$
	\item[iii)] $$\left(\frac{n^k}{n!}\right)_{n\in\N}\:,$$
						wobei $k$ eine feste nat�rliche Zahl sei.
\end{itemize}

\aufgabe{} 
\begin{itemize}
	\item[i)] Sei $z\in\C$. Zeigen Sie: Konvergiert die Folge $(z^n)_{n\in\N}$ gegen ein $w\in\C$, dann gilt $w=zw$.
	\item[ii)] Beweisen Sie, dass eine beschr�nkte Folge, die nicht konvergiert, mehr als einen H�ufungspunkt besitzt.
\end{itemize}

\aufgabe{} Seien $a>0$ und $x_0>0$ gegeben. Wir betrachten die rekursiv definierte Folge
\begin{equation*}
	x_{n+1}=\frac{1}{2}\Big(x_n+\frac{a}{x_n}\Big)\,,
	\label{rek}
\end{equation*}
also $x_1:=\frac{1}{2}\Big(x_0+\frac{a}{x_0}\Big)$, $x_2:=\frac{1}{2}\Big(x_1+\frac{a}{x_1}\Big)$, $x_3:=\frac{1}{2}\Big(x_2+\frac{a}{x_2}\Big)$, usw.
\begin{itemize}
	\item[i)] Beweisen Sie, dass $x_n^2-a\geq0$ f�r alle $n\in\N$. Folgern Sie daraus, dass $x_n\geq\sqrt{a}$ f�r alle $n\in\N$.
	\item[ii)] Zeigen Sie, dass $x_n-x_{n+1}\geq0$ f�r alle $n\in\N$.
	\item[iii)] Folgern Sie aus i) und ii), dass die Folge $(x_n)_{n\in\N}$ konvergiert. Zeigen Sie, dass gilt: $$\lim_{n\rightarrow\infty}x_n=\sqrt{a}\,.$$
\end{itemize}


\end{document}
